\documentclass{article}

\title{Cover Letter For Lynx}
\author{Anthony Shea}
\date{\today}
\begin{document}
\maketitle
    Dear Lynx,
    
    I am writing to express my interest in the System Test/Automation Engineer role.
    I found out about the role from my neighbor Addison Thomas. I studied Systems Engineering at the United States Air Force Academy.
    While there, I learned how to translate user needs into design requirements into specific features.
    I worked as an Intellienge analyst in the Air Force after the academy.
    For the last two and a half years, I have been working with the the job title Data Scientist.
    However, I generally had to build software to deploy the models I created for the companies I worked for.

    For example, at my last company, I had to deploy a convolutional neural network.
    The project was started before I took the job, but I basically had to start from scratch.
    The previous employee had over ten python modules that were all over a thousand lines of code each.
    I was able to get the project to completion and also get the whole project down to around two hundred lines total when I finally deployed the first working version.
    My predecessor had to much code because, after creating the TCP connection, he checked each byte to see if he had the start of a new message and then checked each byte until he had the end of that message.
    Once he did that he put the message in RabbitMQ, then took the message out and parsed it from HDF5 to a python dictionary, then put it back into RabbitMQ, then put it in the ML model, then put it in the database.
    The first thing I did was talk to the data provider and establish that we needed an 8 byte header for the message length.
    Using that I could just ask for 8 bytes get the message length, then ask for the rest of the message. I used a simple queue data structure to store the message and then set up the model as a dedicated TensorFlow server that did inference and then sent the output to the database.

    After that I was able to capture metrics for my machine learning pipeline using beats from the elasticstack and a kibana dashboard.
    I was able to confirm latency and throughput requirements were met using my logs and that dashboard.
    I believe that in future projects I could clearly identify requirements from needs and then develop tests for those requirements to see that the needs are met.
    

\end{document}